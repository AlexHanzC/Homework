\documentclass[fontset=fandol]{ctexart}  
\usepackage{geometry}  
\geometry{margin=1.5cm, vmargin={0pt,1cm}}  
\setlength{\topmargin}{-1cm}  
\setlength{\paperheight}{29.7cm}  
\setlength{\textheight}{25.3cm}  
  
\usepackage{xcolor}  % 使用 xcolor 而非 color  
\usepackage{listings}  
  
% 设置代码样式  
\lstset{  
    basicstyle=\ttfamily\small,  
    keywordstyle=\color{black},  
    commentstyle=\color{gray},  
    stringstyle=\color{red},  
    numbers=left,  
    numberstyle=\tiny\color{gray},  
    stepnumber=1,  
    numbersep=10pt,  
    backgroundcolor=\color{white},  
    showspaces=false,  
    showstringspaces=false,  
    frame=single,  
    rulecolor=\color{black},  
    tabsize=4,  
    captionpos=b,  
    breaklines=true,  
    breakatwhitespace=false,  
    escapeinside={(*@}{@*)}  % 更改逃逸字符设置  
}  

\title{BST\_remove的改善}
\author{汉铮 陈}
\date{\today}

\begin{document}

\maketitle

\section{remove函数的完善}
首先创建辅助函数 findParent(BinaryNode* \&subroot, const Comparable \&x)  用于找到节点element为x的parent,主要是为了后续更新节点用,代码如下:

\begin{lstlisting}[language=C++, caption={findParent()函数}]  
    BinaryNode* findParent(BinaryNode* &subroot, const Comparable &x)
    {
        if (subroot == nullptr || subroot->element == x) return nullptr;
        BinaryNode* parent = nullptr;
        BinaryNode* cur = subroot;
        while (cur != nullptr && cur->element != x)
        {
            parent = cur;
            if (x < cur->element)
            {
                cur = cur->left;
            }
            else
            {
                cur = cur->right;
            }
        }
        if (cur == nullptr) return nullptr;//没有找到
        if ((parent->left == cur || parent->right == cur) && cur->element == x) return parent;
        return nullptr;        
    }
\end{lstlisting}

而remove()函数的实现只需要在原来的基础上,自下而上增加平衡和height更新的操作直到某一节点高度更新前后相同即可,具体代码如下:
\begin{lstlisting}[language=C++, caption={remove函数}]  
    void remove(const Comparable &x, BinaryNode * &t) 
    {
        if(!t) return;//如果t为空节点
        BinaryNode* par = nullptr; // 父节点的指针
        BinaryNode* cur = t;
        while (cur != nullptr && cur->element != x) 
        {
            par = cur;
            if (x < cur->element) cur = cur->left;
            else cur = cur->right;
        }
        if(!cur) return;
        if(!(cur ->left) || !(cur -> right))//要删除的节点至多只有一个child节点
        {
            if(!par)//如果删除根节点
            {
                t = (cur -> left)? cur->left : cur->right;//修改根节点
            }
            if(par -> left == cur)//t是par的左节点
            {
                par -> left = (cur -> left)? cur->left : cur->right;
            }else//cur is parent's right
            {
                par -> right = (cur -> left)? cur->left : cur->right;
            }
            delete cur;
            cur =nullptr;
        }else//有两个child节点
        {
            BinaryNode* min = detachMin(cur->right);
            min -> left = cur -> left;// 将当前节点的左子树链接到后继节点
            min -> right = cur -> right;// 将当前节点的右子树链接到后继节点
            if(!par)//如果删除根节点
            {
                t = min;//修改根节点
            }
            else if(par -> left == cur)//cue是par的左节点
            {
                par -> left = min;// 父节点指向后继节点
            }else//cur是右节点
            {
                par -> right = min;// 父节点指向后继节点
            }
            delete cur;
            cur = nullptr;
        }
        while (par != nullptr) 
        {
            int oldHeight = par->height; // 记录平衡前的高度
            balance(par); // 调整 cur 的平衡
            par->height = max(height(par->left), height(par->right)) + 1; // 更新 cur 的高度
            if (par->height == oldHeight) break; // 如果高度没有变化,提前停止
            par = findParent(t,par -> element);
        }
    }
\end{lstlisting}
完

\end{document}
