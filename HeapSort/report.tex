\documentclass[fontset=fandol]{ctexart}  
\usepackage{geometry}  
\geometry{margin=1.5cm, vmargin={0pt,1cm}}  
\setlength{\topmargin}{-1cm}  
\setlength{\paperheight}{29.7cm}  
\setlength{\textheight}{25.3cm}  
  
\usepackage{xcolor}  % 使用 xcolor 而非 color  
\usepackage{listings}  
  
% 设置代码样式  
\lstset{  
    basicstyle=\ttfamily\small,  
    keywordstyle=\color{black},  
    commentstyle=\color{gray},  
    stringstyle=\color{red},  
    numbers=left,  
    numberstyle=\tiny\color{gray},  
    stepnumber=1,  
    numbersep=10pt,  
    backgroundcolor=\color{white},  
    showspaces=false,  
    showstringspaces=false,  
    frame=single,  
    rulecolor=\color{black},  
    tabsize=4,  
    captionpos=b,  
    breaklines=true,  
    breakatwhitespace=false,  
    escapeinside={(*@}{@*)}  % 更改逃逸字符设置  
}  

\usepackage{booktabs} % 表格增强功能(可选)
\usepackage{float}
\usepackage{array}    % 表格增强功能(可选)


\title{Heap report}
\author{汉铮 陈}
\date{ December 2024}

\begin{document}

\maketitle
\section{简介}
本文主要分为两部分开展,第一部分主要说明堆排序的代码实现,第二部分主要分析比较堆排序的效率。
\section{HeapSort 实现}
对一个序列,无论有序还是无序,对其进行堆排序,大致上分为两步;1,将序列转化为类似于Binary heap的结构 2,对具有堆结构的序列进行排序。

\textbf{基本原理}:不妨假设有一个size为n的序列,由于排序后的序列是递增序列,所以我们首先应当将序列转化为\textbf{大顶堆}(即堆顶元素为当前序列的最大值,在序列中为第一个元素)。然后将该元素与序列最后元素进行\textbf{swap}。此时序列最后一个元素为整个序列的最大元素,为有序状态,因此在后续的排序中可以省去。而堆顶元素为原来序列的最后一个元素,破坏了堆的顺序性质,则需要通过\textbf{percolateDown(下滤)}操作来维护堆的性质。维护完后,将前n-1个元素看作为一个堆,此时只要重复以上步骤便可以得到第二大的元素......以此类推,共n-1次便可以得到一个有序的序列。

\textbf{Remark:}区别于STL中的make\_heap函数和sort\_heap()函数,本文中HeapSort的实现基于容器的下标而非迭代器。
\subsection{Step1 BuildHeap}
在这一部分我们将创建大顶堆,但区别于Binary heap,序列的第一个元素下标为\textbf{0}而非\textbf{1},所以对于创建堆的代码较以前有所不同,主要体现在:下标的为i的元素的左孩子下标为"2*i+1"而非"2*i"

创建大顶堆涉及下滤操作。而下滤的主要原理是:将该位置的元素与其为位置的左右孩子比较,下降到合适的层高。并且与左右孩子中的较大者交换保证了堆顶元素为序列中的最大者。其代码如下:
\begin{lstlisting}[language=C++, caption={percolatdown()函数}]  
template <typename T>
/**
 * @brief 下滤操作
 * @param i 下滤的起点
 * @param n 下滤终点
 */
inline void percolatedown(vector<T> & vec, int i ,int n)
{
    int child = i;
    T tmp;

    for(tmp = std::move(vec[i]) ; 2*i+1 < n ; i = child ) 
    {
        child = 2*child +1;//左孩子
        if(child < n-1 && vec[child] < vec[child +1])
            ++child;//左右孩子中的较大者
        if(tmp < vec[child])
            vec[i]=std::move(vec[child]);
        else break;
    }
    vec[i] = std::move(tmp);
}

\end{lstlisting}

对于一元素数量为n的堆(索引从0开始),它的叶子节点的索引范围是[n/2]到n-1,而叶子不需要进行下滤操作,故只需要对0到[n/2]-1的元素从下往上进行下滤操作即可。具体代码如下:
\begin{lstlisting}[language=C++, caption={BuildHeap()函数}]  
template <typename T>
/**
 * @brief 创建大顶堆,即堆顶为元素最大值
 */
inline void BuildHeap(vector<T> &vec)
{
    for(int i = vec.size()/2-1;i >= 0;--i)
        percolatedown(vec,i,vec.size());
}
\end{lstlisting}


\subsection{Step2 SortHeap}
Sort的原理前文已经简述,其本质上是一个删除最大元素的操作,在每次交换堆顶元素以及堆最后一个元素后,只要将第一个元素下滤即可保持堆的顺序性质,值得注意的是,每找出一个最大元素,堆的大小就减1。具体代码如下:
\begin{lstlisting}[language=C++, caption={Sort()函数}]  
template <typename T>
/**
 * @brief 将堆进行排序
 */
inline void Sort(vector<T> &vec)
{
    for(int i = vec.size()-1 ; i >0 ; --i)
    {
        swap(vec[0],vec[i]);
        percolatedown(vec,0,i);
    }
}
\end{lstlisting}

最终HeapSort的代码为:
\begin{lstlisting}[language=C++, caption={Heapsort()函数}]  
template <typename T>
/**
 * @brief 将堆进行排序
 */
template <typename T>
inline void Heapsort(vector<T> &vec)
{
    BuildHeap(vec);
    Sort(vec);
    return;
}
\end{lstlisting}



\section{HeapSort 测试}
这一部分主要为分为两部分:1,设计检查函数和特定队列 2,比较HeapSort和sort\_heap运行时间 3,分析时间效率
对于检查函数的设计和特定队列的设计并不会着重说明,主要分析后两者。
\subsection{check函数}
check函数主要输出排序的时间和判断排序结果是否正确。对于前者主要通过measureTime函数(利用chrono)记录排序的时间。判断排序正确性,则通过比较具有相同元素的两个初始序列经过HeapSort和sort\_heap(还有make\-heap)后元素是否一一对应。

\textbf{Remark:}只有在同一个序列的情况下,两者比较才有意义。其次比较的是排序的时间,创建堆结构的时间并不再比较范围内。但是判断正确性却是创建堆和排序的共同结果。所以,在check函数中,笔者并没有直接利用HeapSort这个函数,而是用其两个内置函数Sort函数和Buildheap函数。具体代码如下;

\begin{lstlisting}[language=C++, caption={check()函数&measuretime模版函数}]
template <typename Func> // 测量运行时间的模板函数
long long measureTime(Func func, std::vector<int> &data) {
    auto start = std::chrono::high_resolution_clock::now();
    func(data);
    auto end = std::chrono::high_resolution_clock::now();
    return std::chrono::duration_cast<std::chrono::milliseconds>(end - start).count();
}

void check(vector<int> &vec1,vector<int> &vec2)
{
    int flag = 1;//旗帜
    BuildHeap(vec1);
    std::cout << "Heapsort: " << measureTime(Sort<int>, vec1) << " ms\n";
    make_heap(vec2.begin(),vec2.end());
    std::cout << "sort_heap: " << measureTime([](std::vector<int>& data) {std::sort_heap(data.begin(), data.end());}, vec2) << " ms\n";
    
    for(int i = 0; i<vec1.size() ; ++i)//判断正确性
    {
        if(vec1[i] != vec2[i]) 
            cout << "Error!" << endl;
            flag = 0; break;
    } 
    if(flag == 1) cout << "PASS!" << endl;
}
\end{lstlisting}

值得注意的是构建模版函数func接收的是vector<int> \&data,而sort\_heap则接收两个容器的迭代器,故无法直接套用,此处运用了lambda表达式来解决这个问题。

\subsection{序列的构造}
接下来是四种序列的创建,此处不做过多赘述。

随机序列,具体代码如下:
\begin{lstlisting}[language=C++, caption={RandomSeq()函数}]  
const int SIZE = 1000000;

// 创建随机序列
vector<int> RandomSeq() 
{
    vector<int> sequence(SIZE);
    random_device rd;
    mt19937 gen(rd());
    uniform_int_distribution<> dis(1, 1000000);  // 生成1到1000000之间的随机数

    for (int i = 0; i < SIZE; ++i) {
        sequence[i] = dis(gen);
    }
    return sequence;
}
\end{lstlisting}

递增序列,具体代码如下:
\begin{lstlisting}[language=C++, caption={OrderedSeq()函数}]  
// 创建有序序列(递增)
vector<int> OrderedSeq() {
    vector<int> sequence(SIZE);
    for (int i = 0; i < SIZE; ++i) {
        sequence[i] = i + 1;  // 顺序填充1到SIZE
    }
    return sequence;
}
\end{lstlisting}

递增序列,具体代码如下:
\begin{lstlisting}[language=C++, caption={ReverseSeq()函数}]  
// 创建逆序序列(递减)
vector<int> ReverseSeq() {
    vector<int> sequence(SIZE);
    for (int i = 0; i < SIZE; ++i) {
        sequence[i] = SIZE - i;  // 填充SIZE到1
    }
    return sequence;
}
\end{lstlisting}

部分重复序列,具体代码如下:
\begin{lstlisting}[language=C++, caption={RepetitiveSeq()函数}]  
// 创建部分元素重复序列
vector<int> RepetitiveSeq() {
    vector<int> sequence(SIZE);
    random_device rd;
    mt19937 gen(rd());
    uniform_int_distribution<> dis(1, 500000);  // 生成1到500000之间的随机数

    for (int i = 0; i < SIZE; ++i) // 随机生成大部分元素
    {
        sequence[i] = dis(gen);
    }

    uniform_int_distribution<> disRepeat(0, SIZE - 1);// 随机重复一些元素
    for (int i = 0; i < SIZE / 10; ++i) {  // 重复10%的元素
        int index = disRepeat(gen);
        sequence[i] = sequence[index];
    }
    return sequence;
}

\end{lstlisting}

\subsection{main函数及测试结果}

如上文所述,每种类型序列创造后,再复制一遍保证可比较性,此处部分代码有所省略,将以表格的形式呈现输出结果
\begin{lstlisting}[language=C++, caption={main()}]  
int main() {
    // 创建四种序列
    vector<int> randomSeq = RandomSeq();
    vector<int> orderedSeq = OrderedSeq();
    vector<int> reverseSeq = ReverseSeq();
    vector<int> repetitiveSeq = RepetitiveSeq();

    vector<int> oldrandomSeq = randomSeq;
    vector<int> oldorderedSeq = orderedSeq;
    vector<int> oldreverseSeq = reverseSeq;
    vector<int> oldrepetitiveSeq = repetitiveSeq;

    check(randomSeq,oldrandomSeq);
    check(orderedSeq,oldorderedSeq);
    check(reverseSeq,oldreverseSeq);
    check(repetitiveSeq,oldrepetitiveSeq);
    return 0;
}

\end{lstlisting}

以某一次试验结果为例:
\begin{table}[H]
\centering
\begin{tabular}{|c|c|c|c|}
\hline
\textbf{type}   & \textbf{my heapsort time} & \textbf{std::sort\_heap time} &\textbf{Validty}\\ \hline
randomSeq       &44ms                       &45ms    &PASS\\ \hline
orderedSeq      &26ms                       &28ms    &PASS\\ \hline
reverseSeq      &26ms                       &29ms    &PASS\\ \hline
repetitiveSeq   &43ms                       &46ms    &PASS\\ \hline
\end{tabular}
\caption{测试结果}
\label{tab:simple_table}
\end{table}

经过多次测试,发现如图表所示,HeapSort中的Sort函数的排序效率略高于STL中的sort\_heaphan函数。笔者猜想有以下原因:
\begin{enumerate}
    \item std::sort\_heap 是泛型算法,支持任何符合堆概念的容器和迭代器。Heapsort 直接操作 std::vector,没有额外的迭代器间接性,因此可能更快。
    \item 笔者的实现中,percolatedown 和其他函数被标记为 inline,这可能使编译器更容易进行函数内联优化。
    \item 避免了 STL 的额外安全检查和通用性逻辑。
\end{enumerate}

\subsection{时间复杂度分析}
这部分的时间复杂度涉及Buildheap和Sort两部分;

首先来看Buildheap。对一个堆来说,如果化为树,则是一个完全二叉树。则对于一个具有N个节点的完全二叉树来说,总存在一个整数h,使得N在$2^h$和$2^{h+1}-1$之间,其中h为树的高度,最底层h=0.所以对于 the worstcase 的情况,即满二叉树,且对于每一个高度不为0的节点(满二叉树的叶子节点),下滤的次数的大小和其高度一样。对于一个N=$2^{h+1}-1$的序列下滤次数总和最多为$2^{h+1}-1-(h+1)$次,h = [$logN$],故时间复杂度为\textbf{O(N)},边界为2N次比较。

再看Sort部分,在swap堆顶元素和堆尾元素后,堆顶元素要进行下滤,the worstcase是下滤到底部,这样的操作有N-1次,由于堆的大小在不断减小,则比较次数的上界显然为$NlogN$,实际上因该为$NlogN-o(N)$其中$o(N)$为$NlogN$的高阶无穷小。

综上对于一个有N个元素随机序列,它的时间复杂度应该与$O(NlogN)$比较。

\end{document}
